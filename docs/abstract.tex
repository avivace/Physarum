\begin{abstract}
We introduce and define the Cellular Automata mathematical model, used to describe the evolution of discrete systems. We give an overview of its relevancy and applications in different fields.

We use this tool to model the "intelligent" behavior of \textit{Physarum polycephalum}, a slime mould extensively studied for its interesting characteristics:

Despite the protist not having a nervous system, it has the capacity to solve computational challenges such as the Shortest Path and instances of the Transportation Problem.
It also exhibits a form of memory and creates efficient networks when given more than two food sources, being able to dynamically re-allocate itself to maintain constant levels of different nutrients simultaneously.

Our main work has been creating an improved Physarum model that performs more similarly to the real mould than the current proposed approximations offered by the scientific community.
Our experimental model closely follows the real mould behaviour without violating the Celluar Automata definition, returning interesting results often more realistic than the other public models.
We proceed to study the global behaviour and topologies that raise from our Cellular Automata rules, applying the simulation on several maps and finally discussing the results and the possible improvements.

We proceed to build a cross-platform software framework to run and visualize a simulation of the described model using modern tools. A UI exposes a series of control features, letting the user monitor and control the simulation.

\end{abstract}
\addtocounter{page}{-1}