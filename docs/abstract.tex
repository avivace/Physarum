\begin{abstract}
We introduce and define the Cellular Automata mathematical model, used to describe the evolution of discrete systems. We give an overview of its relevancy and applications in different fields.

We use this tool to model the "intelligent" behavior of \textit{Physarum polycephalum}, a slime mould extensively studied for its interesting characteristics:

Despite the protist not having a nervous system, it has the capacity to solve computational challenges such as the Shortest Path and instances of the Transportation Problem.
It also exhibits a form of memory and creates efficient networks when given more than two food sources, being able to dynamically re-allocate itself to maintain constant levels of different nutrients simultaneously.

We proceed to build a cross-platform software framework to run and visualize a simulation of the described model using modern tools. A UI exposes a series of control features, letting the user monitor and control the simulation.

\end{abstract}
\addtocounter{page}{-1}