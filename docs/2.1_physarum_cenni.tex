
\section{Physarum: cenni}
\label{physarum_cenni}
Il Physarum Polycephalum è un mixomiceto melmoso unicellulare appartenente al clade Amoebozoa, comunemente conosciuto come fungo mucillaginoso o muffa melmosa, che prospera in condizioni ambientali favorevoli, in particolare quando si creano le giuste combinazioni di umidità, temperatura e presenza di nutrienti.
\par
Se le condizioni non sono adeguate allo sviluppo, il Physarum si comporta come un organismo unicellulare che non dimostra capacità organizzative. In condizioni opportune, si aggrega a creare reti filamentose particolarmente efficienti nella distribuzione fisica.
\par 
Un numero sempre maggiore di ricercatori ha condotto studi riguardanti il Physarum per esplorare la sua intelligenza nell'ambito dei problemi di ottimo. Ciò è avvenuto per via di alcuni esperimenti biologici osservati in condizioni di laboratorio in cui la muffa ha esibito capacità straordinarie, ad esempio calcolando la distribuzione più efficiente della rete ferroviaria di Tokyo o esplorando semplici labirinti trovando la via più breve. 
\par
Per gran parte del ciclo vitale il Physarum si trova sotto forma di un'unica massa citoplasmatica, detta plasmodio, una cellulare plurinucleare costituita da reti di vene protoplasmatiche che arriva a contenere più di 100 000 nuclei.
Durante questa fase l'organismo cerca cibo e il plasmodio circonda il suo cibo e secerne enzimi per la digestione.
Se le condizioni ambientali portano il plasmodio ad essiccare durante la nutrizione o la migrazione, il Physarum forma uno sclerozio. Lo sclerozio è fondamentalmente tessuto indurito multinucleato che serve per proteggere il Physarum per lunghi periodi di tempo, per poi ritornare ad una normale funzionalità una volta che le condizioni favorevoli si ripresentano. In tal caso il plasmodio riappare per proseguire la sua ricerca di cibo.
Come l'approvvigionamento di cibo termina, il plasmodio smette di alimentarsi e inizia la fase riproduttiva. Gambi di sporangi nascono dal plasmodio, è all'interno di queste strutture, che si verifica la meiosi e le spore si formano. Gli sporangi si formano all'aperto in modo che le spore che rilasciano saranno diffuse da correnti di vento.
Le spore possono rimanere latenti per anni se necessario. Tuttavia, quando le condizioni ambientali sono favorevoli per la crescita, le spore germinano e rilasciano sciami di cellule o flagellate o ameboidi (fase di motilità); le cellule brulicanti poi si fondono per formare un nuovo plasmodio. 





NB da verificare: la muffa viene chiamata “intelligente”. In realtà il comportamento della muffa non è intelligente, bensì emergente, e l’emergenza è quello che succede quando un sistema interconnesso di elementi relativamente semplici si autoorganizza dando luogo a un comportamento più intelligente e più adattivo.

NB da verificare: Era parere comune della comunità scientifica presumere l’esistenza di una tipologia di cellula “segna passi”, capace di trovare le risorse e di “comandare” le altre cellule ad aggregarsi. le cellule della Physarum polycephalum potevano essere capaci di aggregarsi seguendo i cambiamenti ambientali e rispondendo alle nuove condizioni senza aver bisogno di una cellula “segna passi”. Così essi svilupparono un modello elegante capace di riprodurle l’organizzazione dell’organismo soltanto attraverso la secrezione da parte di ogni qualunque cellula di una sostanza chimica quando si trova in vicinanza di cibo. La semplice reazione delle altre cellule al ricevimento del segnale chimico era abbastanza per far emergere un comportamento intelligente di organizzazione del gruppo, non possibile da raggiungere dalle singole cellule. 

