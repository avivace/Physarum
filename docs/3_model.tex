Consider only the plasmodium stage of its life cycle, there is no single model that can describe exactly the behavior of Physarum. So far, there is a variety of modeling approaches which also are implemented by a variety of tools.
\par
In this project the behaviour of slime mould during the laboratory experiments have been simulated by a model based on Cellular Automata. Using CA can be justified by the emergence of global behaviour from local interactions, a rule that applies also on the real slime mould.
In particular, the model is based on the representation of diffusion of chemical attractants by NSs and the attraction of the plasmodium, which initiates its exploration from the starting point (SP), by these chemicals. 
\par
All of the biological studies indicated above, like solving a maze and designing a transport network, have observed the optimization behaviour of the plasmodium with the experimental setups roughly consisting of three steps \cite{shirakawa2015construction}:
\begin{itemize}
	\item First step: the plasmodium fully searches the given space
	\item Second step: some sources of attractant or repellent stimuli are given to the plasmodium that is fully and homogeneously spreading in the space
	\item Third step: the plasmodium optimizes the connection between the sources
\end{itemize}

The majority of the studies focus on the behaviour of the plasmodium in a closed space. However, few studies have been done for the plasmodium that is exploring an open space. We believe that we can find more enhanced biological characteristics of the biological entities by observing the behaviours of them in an open and unknown environment. In this study, we thus tried to investigate the exploratory behaviour of the plasmodium in an open space and to understand how the organism makes its decision
in exploration. 

\par
In the proposed CA model \cite{Tsompanas2016} the plasmodium is first starved and then introduced and then introduced into a specific place in the space. Moreover, some NSs which produces chemo-attractants are located at characteristic points. 
In order to simulate this biological experiment, the area is divided into a matrix of squares with identical areas and each square of the surface is represented by a CA cell.
\par
The plasmodium explores the available area, encapsulates the NSs and creates a tubular network that connects all these NSs by a nature-inspired, cost effective and risk avoiding manner. To note the geometry of the network created by the plasmodium depends on the positions of the NSs. Moreover, further parameters can play key role in determining exact structure of plasmodium network.



\par
The model imitates the entire area that it is used in a laboratory experiment using the plasmodium of Physarum. The entire area can be defined, without loss of generality, as a square grid divided into identical square cells that constitutes a set defined as E. This area can be categorized as available area (a set of cells defined as A) and unavailable area (a set of cells defined as U) for the development of the plasmodium.
Also some cells that are included in the available area set of cells, represent the oat flakes that are considered as NSs for the plasmodium (a set of cells defined as N) and one cell represents the place where the plasmodium is initially introduced to the
experimental environment or the SP (a set of one cell defined as S). 
The neighbourhood type used for the proposed model is Moore neighbourhood and the state of the $c_{(i, j)}$ cell at time step t ($ ST^t_{(i, j)}$) is defined as:
\begin{align}
ST^t_{(i, j)} = [AA_{(i, j)}, PM^t_{(i, j)}, CHA^t_{(i, j)}, TE^t_{(i, j)}]
\end{align}
where:

\begin{itemize}
	\item $AA$ stands for Available Area for the plasmodium to explore. It assumes a boolean value:  	
\[AA_{(i, j)}=\begin{cases} \mbox{True}, & \forall i, j: c_{(i,j)} \in A \\ \mbox{False}, &  \forall i, j: c_{(i,j)} \in U\end{cases}\]
	\item $PM$ (Physarum Mass) is a floating-point variable. It indicates the volume of the cytoplasmic material of the plasmodium located on a specific cell
	\item $CHA$ (CHemoAttractant) is a floating-point variable. It represent the concentration of chemo-attractants that are located on a specific cell
	\item $TE$ stands for Tube Existence and represents the participation of a cell in the tubular network inside the body of the slime mould
\end{itemize}

The initial values for parameters $PM$ and $CHA$ are defined as:
\begin{align}
PM^t_{(i, j)}=\begin{cases} 100, & \forall i, j: c_{(i,j)} \in S \\ 0, & \mbox{else}\end{cases}
\end{align}
\begin{align}
CHA^t_{(i, j)}=\begin{cases} 100, & \forall i, j: c_{(i,j)} \in N \\ 0, & \mbox{else}\end{cases}
\end{align}

Taking into consideration the assumptions made for the way the Physarum develops through an available area, which were confirmed by laboratory experiments, it is determined that the plasmodium is "amplified" at a NS and then searches for other NSs, considering the recently encapsulated NS as a new SP. Also, when a NS is covered by the plasmodium, the generation of chemoattractant substances is ceased.
In the proposed model the NSs are turned into SPs when the plasmodium encapsulates them with a sufficient amount of mass. Furthermore, it is realized that the plasmodium is propagating away from the most recently captured NS by taking a semi-circular form. 
\par
The model is analyzed in the following flowchart: TODO inserire flowchart.
\par
The initialization step includes the definition of parameters that have a great impact on the results of the model. These parameters include:
\begin{itemize}
	\item The length of the CA grid
	\item The parameters for the diffusion equation for the cytoplasm of the plasmodium (PMP1, PMP2)
	\item The parameters for the diffusion equation of the  (CAP1, CAP2)
	\item The consumption percentage of the chemo-attractants substances by the plasmodium (CON - Consumption)
	\item The attraction of the slime mould by chemoattractant substances (PA - Physarum Attraction)
	\item The threshold of Physarum Mass that encapsulates a NS (ThPM).
\end{itemize}
\par
After the initialization and for 50 time steps, diffusion equations are used to calculate the values for $CHA$ and $PM$ for every cell in the grid. Every cell uses the values of its neighbours at time step t to calculate the value of the $CHA$ and $PM$ parameter for time step t + 1. 
\par
The contribution to the diffusion of the Physarum Mass of the von Neumann neighbours ($PMvNN$) of the $c_{(i,j)}$ cell is defined as:
\begin{equation}
\begin{split}
PMvNN^t_{(i, j)} = 
(1 + PA^t_{(i, j),(i-1, j}) \times PM^t_{(i-1, j)} - AA_{(i-1, j)} \times PM^t_{(i, j)} +
\\(1 + PA^t_{(i, j),(i, j-1}) \times PM^t_{(i, j-1)} - AA_{(i, j-1)} \times PM^t_{(i, j)} +
\\(1 + PA^t_{(i, j),(i+1, j}) \times PM^t_{(i+1, j)} - AA_{(i+1, j)} \times PM^t_{(i, j)}  +
\\(1 + PA^t_{(i, j),(i, j+1}) \times PM^t_{(i, j+1)} - AA_{(i, j+1)} \times PM^t_{(i, j)}
\end{split}
\end{equation}
Moreover, the contribution to the diffusion of the Physarum Mass of the exclusively Moore neighbours ($PMeMN$) of the $c_{(i,j)}$ cell is defined as:
\begin{equation}
\begin{split}
PMeMN^t_{(i, j)} = 
(1 + PA^t_{(i, j),(i-1, j-1}) \times PM^t_{(i-1, j-1)} - AA_{(i-1, j-1)} \times PM^t_{(i, j)} +
\\(1 + PA^t_{(i, j),(i+1, j-1}) \times PM^t_{(i+1, j-1)} - AA_{(i+1, j-1)} \times PM^t_{(i, j)} +
\\(1 + PA^t_{(i, j),(i-1, j+1}) \times PM^t_{(i-1, j+1)} - AA_{(i-1, j+1)} \times PM^t_{(i, j)}  +
\\(1 + PA^t_{(i, j),(i+1, j+1}) \times PM^t_{(i+1, j+1)} - AA_{(i+1, j+1)} \times PM^t_{(i, j)}
\end{split}
\end{equation}

The total $PM$ for a cell $c_{(i,j)}$ for time t + 1 is a sum of the contributions of its neighbours with appropriate weights and is defined as:
\begin{equation}
PM^{t+1}_{(i, j)} = PM^t_{(i, j)} + PMP1 \times [PMvNN^t_{(i, j)} + PMP2 \times PMeMN^t_{(i, j)}]
\end{equation}

If a neighbouring cell is representing unavailable area, there is no contribution to the diffusion. 
\par
The parameter $PA_{(i, j),(k,l)}$ represents the attraction of the $PM$ in cell $c_{(i,j)}$ towards the direction of an adjacent cell $c_{(k,l)}$ ,modeling the attraction of the organism towards the higher gradient of chemoattractants. 
\par
It is equal to a predefined constant (PAP) for the neighbour with the higher value of chemoattractant and equals to the negative value of the parameter PAP for the neighbour across the neighbour with the higher value of chemoattractant. For all the other neighbours the parameter PAP is equal to zero. The $PA$ parameter for cell $c_{(i,j)}$ towards its north neighbour $c_{(i-1,j)}$ is defined as:
\begin{equation}
PA^t_{(i, j),(k,l)}=
\begin{cases} 
PAP, & if CHA_{(i-1, j)} = MAX(CHA_{(k, l)} \forall k, l: i - 1 \leq k \leq i + 1 \\and j - 1 \leq l \leq j+1) \\ 
- PAP, & if CHA_{(i-1, j)} = MAX(CHA_{(k, l)} \forall k, l: i - 1 \leq k \leq i + 1 \\and j - 1 \leq l \leq j+1) \\ 
0, & \mbox{else}
\end{cases}
\end{equation}

\par
The contribution to the diffusion of the chemoattractants for the plasmodium of the von Neumann neighbours ($CHAvNN$) of the $c_{(i,j)}$) cell is defined as:
\begin{equation}
\begin{split}
CHAvNN^t_{(i, j)} = 
(CHA^t_{(i-1, j)}) - AA_{(i-1, j)} \times CHA^t_{(i, j)} +
\\(CHA^t_{(i, j-1)}) - AA_{(i, j-1)} \times CHA^t_{(i, j)} +
\\(CHA^t_{(i+1, j)}) - AA_{(i+1, j)} \times CHA^t_{(i, j)}  +
\\(CHA^t_{(i, j+1)} - AA_{(i, j+1)} \times CHA^t_{(i, j)}
\end{split}
\end{equation}

Moreover, the contribution to the diffusion of the chemoattractants for the plasmodium of the exclusively Moore neighbours ($CHAeMN$) of the $c_{(i,j)}$ cell is defined as:

\begin{equation}
\begin{split}
CHAeMN^t_{(i, j)} = 
(CHA^t_{(i-1, j-1)}) - AA_{(i-1, j-1)} \times CHA^t_{(i, j)} +
\\(CHA^t_{(i+1, j-1)}) - AA_{(i+1, j-1)} \times CHA^t_{(i, j)} +
\\(CHA^t_{(i-1, j+1)}) - AA_{(i-1, j+1)} \times CHA^t_{(i, j)}  +
\\(CHA^t_{(i+1, j+1)}) - AA_{(i+1, j+1)} \times PM^t_{(i, j)}
\end{split}
\end{equation}

As a result, the total $CHA$ for a $c_{(i,j)}$ cell for time t + 1 is defined as:
\begin{equation}
CHA^{t+1}_{(i, j)} = CON \times {CHA^t_{(i, j)} + CAP1 \times (CHAvNN^t_{(i, j)} + CAP \times CHAeMN^t_{(i, j)})}
\end{equation}

\par
The multiplication with the parameter CON provides the imitation of the consumption of the chemoattractant substances by the plasmodium.
\par
Also in this case as in the diffusion of $PM$, if a neighbouring cell is representing unavailable area there is no contribution to the diffusion.
\par
After every 50 time steps of calculating the diffusion equations in the available area, if any NS is covered with over the predefined PM (ThPM) it is connected with a SP with a Tube that follows the gradient of the PM to the higher value. More specifically, starting from the cell representing the encapsulated NS, the adjacent cell with the higher PM value is selected to participate to the tubular network. Then the cell selected to participate to the tubular network selects the next cell from its neighbours with the higher PM value to participate to the tubular network and so on, until a SP is reached.
Finally, this NS will be transformed to a SP (PM = 100) and will act as a SP for the remaining time steps as illustrated in the following equations:

\begin{equation}
PM^t_{(i, j),(k,l)}=
\begin{cases} 
0, & \forall i, j: c_{(i,j)} \in U \\ 
100, & \forall i, j: c_{(i,j)} \in S \\ 
100, & \forall i, j: c_{(i,j)} \in N and  PM^t_{(i, j)} \geq ThPM 
\end{cases}
\end{equation}

\begin{align}
CHA^t_{(i, j),(k,l)}=
\begin{cases} 
100, & \forall i, j: c_{(i,j)} \in N and PM^t_{(i, j)} < ThPM\\ 
0, & \forall i, j: c_{(i,j)} \in N and  PM^t_{(i, j)} \geq ThPM 
\end{cases}
\end{align}

If more NSs are covered with the predefined PM, they are connected to the nearest SP and they are all transformed to SPs.










\par
DA INSERIRE ?: A logical question that comes to mind is why procedures (2) to (5) are executed
for a second time. As identified in some laboratory experiments [44], after a seemingly
random and not certain amount of time the plasmodium seems to change the
formation of its protoplasmic networks and abandon some NSs. Then it seems to
regenerate in a manner and re-colonize some NSs, meaning it forms new tubular
edges that connect NSs that were already connected to other NSs. As the CA model
is designed without the use of probabilistic equations, it uses a second starting point
to regenerate and explore the available area once more. That point will be a point of
interest (NS), which is empirically chosen to be away from the initial SP
\par
DA INSERIRE ?: The second to last NS to
be encapsulated was chosen, based on the fact that it is far enough from the initial
SP and it is less likely to be a point of interest surrounded by unavailable area that
would cause difficulties for the growth of the plasmodium. Finally, the time period
of 50 time steps for the diffusion equations, was also empirically chosen, although
the alternation of that will cause little difference to the results of the model.

