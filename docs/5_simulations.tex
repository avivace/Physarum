The most thoroughly studied laboratory experiment that the plasmodium of Physarum is subjected to, is the imitation and optimization of human-made transport networks.

Adamatzky proposed another approach of the same problem. The main difference can be found in the
initial conditions. Adamatzky placed the plasmodium in
one place of the maze and, simultaneously, placed one FS
in another place of the maze, before the plasmodium
covers all the maze. The biological experiments show that
the plasmodium spreads its pseudopodia trying to reach
the food. Simultaneously, the food, releases the
chemo-attractants to any direction in the maze. When the
plasmodium finds those chemo-attractants, it follows
them to the source food forming the minimum distance
path between its initial site and the food site. So the
plasmodium solves the maze in one pass because it is
assisted by a gradient of chemo-attractants propagating
from the target food. This approach, is modeled in this
paper.