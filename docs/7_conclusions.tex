In this project we presented Physarum polycephalum from a biological point of view and explained complex patterns observed in this puzzling organism. For our purposes of modeling we introduced and defined the Cellular Automata mathematical model  which can perfectly describe slime mould's distributed control because of its properties.

\par
After reviewed the literature on several models, we have decided to use the CA based model of Tsompanas et al. \cite{Tsompanas2016} that mimics the foraging strategy and tubular network formation. In particular, the model is based on the representation of diffusion of chemical attractants by NSs and the attraction of the plasmodium, which initiates its exploration from the starting point (SP), by these chemicals. 

\par
However this model seemed not to follow the true behavior of slime mould and for this reason we have proposed our experimental model - based on this well-known work - which objective was to fix the issues that emerged from the many executions of the original model.

\par
In particular, their algorithm reached the results for the different types of simulation neglecting important realistic constraints. We found their CA's behaviour an excessive approximation of the real mould dynamics, therefore we worked on improving their original algorithm to the point of changing most of the steps.

\par
The several limitations of \cite{Tsompanas2016} model showed that more technical information should be provided by the authors to completely reproduce the results they claim to have obtained.
In any case our experimental model showed a much more realistic behaviour, loyally following the internal dynamics of the mould system. This created a more solid algorithm that can succed in all the different testing enviroments. The results of the Minimum Spanning Tree and Maze Solving are suboptimal but with a negligible error from the mathematical solution.

\par
Our framework provided an easy way to compile and deploy the simulations, allowing an easy tuning of the parameters for observation purposes.

\section{Future developments}
We have drawn a map with two SPs to see the behavior. However, the model is based on a single Physarum at the initial state and therefore uses a single gradient for the creation of the tubes. Therefore it is normal that with two SPs the computation is not successful in these circumstances.

\par
The models based on single Physarum as those cited in the previous chapter addressed only attraction force of just one Physarum towards a food resource.

\par
So a plausible future development concerns the possibility of considering a Physarum competitive behaviour where a group of Physarum with different masses and motivation - hunger and satiety - each having autonomous behaviours react to each other and their own local environment. 

\par
This type of model has already been proposed \cite{hex_phy} and it considered other forces acting on Physarum based on metaheuristics inspired from Physarum behaviour in a competition. They assume that competing Physarum will exert repulsion forces on each other which will affect the evolution of the whole system and so they created a new formula to compute two forces acting on Physarum: 

\begin{itemize}
	\item the chemo attraction force based on the combination of chemical mass and chemical quality
	\item the repulsion negative forces that competing Physarum exert on each other
\end{itemize}

Chemo attraction forces exerted on Physarum will be a function of food resource (with different mass and quality) and Physarum hunger motivation. If Physarum is satisfied, it would appreciate the quality of chemical rather than the mass, and if it is hungry, vice versa.

\par
Another important improvement that could be made in our experimental model is reversing the current shrinking phase of the mould: our current local rules of the Cellular Automata move the disperesed mass of the Physarum towards the tubes in a unrealistic order as the furthest cells are considered only after the others. This creates temporary unrealistic holes in the mould body. We are confident that better local rules can be made for the shrinking process following the change of the gradient of the real Physarum.


