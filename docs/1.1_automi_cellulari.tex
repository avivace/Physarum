\section{Automi cellulari}
\label{automi_cellulari}

Un Automa Cellulare è un modello matematico discreto e dinamico in grado di immagazzinare ed elaborare informazione utilizzando una serie di regole.
\par
L’idea di base degli Automi Cellulari è quella della simulazione di comportamenti complessi derivanti da sistemi naturali.
\par
Dal punto di vista teorico, gli Automi Cellulari vengono descritti e definiti negli anni cinquanta da John Von Neumann e Stanislaw Ulam, che li usano per simulare processi naturali e biologici. La parte interessante di questo processo non è solo la simulazione come imitazione di comportamenti biologici, ma la stessa creazione di modelli automatici in grado di indagare la vera natura di sistemi complessi esistenti in natura. In questo modo, gli automi cellulari diventano uno strumento per studiare l’auto-organizzazione e l’emergenza nei sistemi complessi.
\par
Un Automa Cellulare consiste di una griglia di dimensione finita costituita da un numero anch’esso finito di celle. Ciascuna cella può assumere un insieme finito di stati \texttt{k} - con \texttt{k} maggiore o uguale a 2 - e l'aggiornamento avviene in tempo discreto. All’istante t = 0 si assegna ad ogni cella un determinato stato e l’insieme di questi stati costituisce la configurazione iniziale dell'automa cellulare. Ad ogni istante temporale (solitamente avanzando t di 1) ogni cella si aggiorna e una nuova generazione, seguendo le regole - predeterminate ed invarianti nel tempo - del sistema, sostituisce la generazione precedente. Il cambiamento di stato di ogni singola cella è determinato solamente dal proprio stato attuale e dagli stati delle celle vicine a quella attuale. 
\par
Considerando gli Automi Cellulari bidimensionali, ci sono due metodi considerati per la definizione dei \textit{vicini}:
\begin{itemize}
\item Von Neumann, che considera il vicinato l'insieme delle celle in cui è compresa la cella al centro e le celle posizionate a Nord, Est, Sud e Ovest rispetto a quella centrale
\item Moore, che considera il vicinato l'insieme delle celle del metodo Von Neumann a cui vanno aggiunte le celle posizionate a Nord-Est, Sud-Est, Sud-Ovest e Nord-Ovest rispetto a quella centrale
\end{itemize}
\par
Un caso particolare è \texttt{The Game of Life}, un Automa Cellulare creato sul finire degli anni sessanta dal matematico John Horton Conway con l’intento di produrre un modello semplice dell’idea di Von Neumann della macchina capace di riprodurre se stessa avendo le proprietà della Macchina Universale di Turing. In \texttt{The Game of Life} le celle, interagendo con le otto celle vicine, possono avere due stati: vive o morte.
\par
Questo comportamento viene definito attraverso quattro regole:
\begin{itemize}
\item ogni cellula che ha meno di due vicini vivi, muore per solitudine
\item ogni cellula con due o tre vicini vivi, sopravvive alla prossima generazion
\item ogni cellula con più di tre vicini vivi, muore di sovraffollamento
\item ogni cellula morta con tre vicini vivi, nasce attraverso la riproduzione
\end{itemize}
\par
Il sistema nasce da una prima generazione chiamata \textit{seme} che può avere qualunque configurazione iniziale. Ad ogni istante temporale le quattro regole vengono applicate a tutte le cellule simultaneamente facendo morire, sopravvivere o nascere la prossima generazione. Il fenomeno che genera l’interesse su \texttt{The Game of Life} è che un processo così semplice e deterministico è capace di creare un comportamento che sta al limite dell’insorgenza del caos, capace di far emergere strutture estremamente complesse che possono avere anche significati spaziali.
