An increasing number of researchers have conducted studies concerning the Physarum to explore its intelligence in the field of optimal problems. This happened due to some biological experiments observed in laboratory conditions in which the Physarum exhibited complex patterning, adaptive behavior and extraordinary capacities to build efficient networks.
\par
It is very important to remember that the slime mould is a puzzling organism because it possesses no neural tissue yet, despite this, are known to exhibit complex biological and computational behavior: it is not explicitly trying to solve computational problems. So the idea served by several researchers was how to take advantage of Physarum’s to survive in order to solve complex problems. 
\par
In the laboratory experiments that demonstrate these computing capabilities, small food sources - agar blocks containing ordinary oat flakes - are presented at various positions to a starved plasmodium, it endeavours to reach them all; as a consequence, only a few tubes are in contact with each individual food source.  The organism attempts to optimize the shape of the network to facilitate effective absorption of the available nutrients. However, this might be difficult to achieve when multiple food sources are presented because of the limited body mass of the organism.
\par
This network shape of the body enables certain physiological requirements to be met \cite{nakagaki2004obtaining}:
\begin{itemize}
	\item absorption of nutrients from food souces as efficiently as possible because almost all the body mass stays at the food sources to enable absorption
	\item maintenance of the connectivity and intracellular communication throughout the organism
	\item meeting the constraint of limited resource of body mass. The network shape is regarded as a solution for the
organism’s survival problems.
\end{itemize}

Contrary to this, if food is plentiful, the organism finally splits into two pieces on two food sources.

\par
The organism requires a well hydrated substrate. All true slime moulds reproduce by sporulation. Certain factors, such as starvation, light irradiation and dehydration will prompt the plasmodium to irreversibly transform into a multitude of black, globulose structures known as sporangia that harbour the organism’s spores.
\par
The name Physarum refers to the fact that multiple apparently autonomous leading edges may exist in one plasmodium. This is an observation of note as some of the first work on slime mould was based on the principle that Physarum can "choose" the most efficient path between food sources.
\par
The biological basis for this involves the Physarum identifying chemical gradients with multiple advancing margins before deciding to navigate along the strongest gradient. This has been interpreted as slime mould undertaking problem solving and network optimization, such as demonstrated in the following ground-breaking experiments.


\paragraph{Maze solving}
Nakagaki et al. \cite{nakagaki2000intelligence} were the first to observe that the plasmodium of the slime mould changes its shape as it crawls over a plain agar gel. If food is placed in two certain spots, it puts out pseudopodia that connect those food spots. The most interesting part is that the plasmodium had the ability to find the minimum-length solution between two points in a maze. This happens because Physarum reduces its mass, from the paths of the maze that is far from the minimum distance, and strengthens its tubes that belong to the minimum distance.
\paragraph{Network formation}
In 2010 Tero et al. \cite{Tero439} compared the actual rail network in Japan with a Physarum network consisted by 36 nutrients sources (NSs) that represented the geographical locations of cities in Tokyo area. The Physarum was planted on Tokyo and from there started its foraging  and exploration for NSs until it filled much of the available land space. Then the organism started to concentrate on the NSs by thinning out the network to leave a subset of larger interconnecting tubes. The topology of many Physarum networks appeared similar to the rail network. The conclusion was that Physarum networks showed characteristics with comparable qualities to those of the rail network in terms of cost, transport efficiency and fault tolerance.\\

\par
AGGIUNGERE MODELLO MATEMATICO?
\par

\par
SEZIONE SEPARATA?
\par
In the recent years, computer scientists have been inspired by biological systems for computational approaches, in particularly with respect to complex optimization and decision problems \cite{grube2016physarum}. In this context, Physarum emerged as a model organism which has attracted substantial interest. The aforementioned experiments require expensive and specialized equipment and some experience on basic biological laboratory techniques. However, the majority of scientists are unfamiliar with such methods and the experiments on a living organism may last a lot of hours or maybe some days to provide data.
\par
A commonly proposed alternative alleviating these difficulties is software models that simulate the behavior of the plasmodium and provide similar results. The advantages of a software model over the real slime mould are repeatability and faster productions of results.
\par
As slime mould has no brain or any central processing system, its distributed control can be perfectly described by the local rule of CAs. It is noteworthy that CAs are known for emerging global behavior from local interactions. The CA model was firstly proposed to design high-quality networks \cite{gunji2008minimal}. In this model, the optimization process is consistent with the properties of real cells. 
