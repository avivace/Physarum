An increasing number of researchers have conducted studies concerning the Physarum to explore its intelligence in the field of optimal problems. This happened due to some biological experiments observed in laboratory conditions in which the Physarum exhibited extraordinary capacities to build efficient networks.
\par
In the laboratory experiments that demonstrate these computing capabilities, slime mould is first starved and then introduced to an area with attractants placed on key positions. The preferred nutrient source is ordinary oat flakes. The organism requires a well hydrated substrate. All true slime moulds reproduce by sporulation. Certain factors, such as starvation, light irradiation and dehydration will prompt the plasmodium to irreversibly transform into a multitude of black, globulose structures known as sporangia that harbour the organism’s spores.
\par
The name Physarum refers to the fact that multiple apparently autonomous leading edges may exist in one plasmodium. This is an observation of note as some of the first work on slime mould was based on the principle that Physarum can 'choose' the most efficient path between food sources.
\par
The biological basis for this involves the Physarum identifying chemical gradients with multiple advancing margins before deciding to navigate along the strongest gradient. This has been interpreted as slime mould undertaking problem solving and network optimization, such as in the ground-breaking experiments that demonstrated Physarum calculating efficient routes between   two selected points in a maze \cite{nakagaki2000intelligence} and approximating transport networks with comparable qualities to those of Tokyo rail system \cite{Tero439}.
\par
In the recent years, computer scientists have been inspired by biological systems for computational approaches, in particularly with respect to complex optimization and decision problems \cite{grube2016physarum}. In this context, Physarum emerged as a model organism which has attracted substantial interest. The aforementioned experiments require expensive and specialized equipment and some experience on basic biological laboratory techniques. However, the majority of scientists are unfamiliar with such methods. 
\par
A commonly proposed alternative alleviating these difficulties is software models that simulate the behavior of the plasmodium and provide similar results. The advantages of a software model over the real Physarum are repeatability and faster productions of results.
\par
As slime mould has no brain or any central processing system, its distributed control can be perfectly described by the local rule of CAs. It is noteworthy that CAs are known for emerging global behavior from local interactions. The CA model was firstly proposed in 2008 to design high-quality networks \cite{gunji2008minimal}. In this model, the optimization process is consistent with the properties of real cells. 