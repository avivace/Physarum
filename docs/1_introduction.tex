\section{Cellular Automaton}
\label{ca}

A cellular automaton is a discrete dynamical system consisting of cells that change their states simultaneously according a local update rule. This update process is repeated at discrete time steps \cite{canotes}.

\paragraph{Formal definition}

Let:
\begin{itemize}
	\item $\mathds{Z}^{d}$ be a $d$-dimensional cellular space, with $d \in \mathds{N}^{+}$ 
	\item $a$

\end{itemize}



\section{Cellular Automata}
\label{automi_cellulari}

A Cellular Automaton (abbrev. CA) is a discrete and dynamic mathematical model capable of storing and processing information on a set of rules.
\par
The basic idea of CAs is the simulation of complex behaviors deriving from natural systems.
\par
The concept was originally described and defined in the 1940s by Stanislaw Ulam and John von Neumann, that use them to simulate natural and biological processes. The interesting part of this process is not only the simulation as an imitation of biological behaviors, but the creation of automatic models that can investigate the true nature of complex systems existing in nature. In this way, cellular automata become a tool to study self-organization and emergence in complex systems.
\par
A CA consists of a regular grid of cells. Each cell can take a finite number \texttt{k} of different states, where \texttt{k} is a number equal or greater than 2. Cells update their states in discrete time. The time step t = 0 is usually considered as the initial step and a given state is assigned to each cell: the set of these states constitutes the initial configuration of the cellular automaton.
At each time step (usually advancing t of 1) each cell is updated and following the invariant over time system's rules a new generation replaces the previous generation. The state change of each single cell is determined only by its current status and by the states of the cells close to the current one.
\par
Regarding the two-dimensional CAs, there are two fundamental types of neighbourhoods that are mainly considered:
\begin{itemize}
\item von Neumann neighbourhood, that consists of the central cell, whose condition is to be updated, and the four cells located to the north, south, east and west of the central cell
\item Moore neighbourhood, that consists of the same cells with the von Neumann neighbourhood together with the four other adjacent cells of the central cell (the north-west, north-east, south-east and south-west cells)
\end{itemize}
\par
The most known CA in the scientific literature is \texttt{the Game of Life}, devised by the John Horton Conway in 1970 with the intention of producing a simple model of von Neumann's idea of the machine capable of reproducing itself and simulate a Turing machine. 
\par
In the universe of \texttt{the Game of Life} each cell is in one of two possible states, alive or dead (or populated and unpopulated, respectively). Every cell interacts with its eight neighbours, which are the cells that are horizontally, vertically, or diagonally adjacent. At each step in time, the following transitions occur:
\begin{itemize}
\item Any live cell with fewer than two live neighbours dies, as if by underpopulation
\item Any live cell with two or three live neighbours lives on to the next generation
\item Any live cell with more than three live neighbours dies, as if by overpopulation
\item Any dead cell with exactly three live neighbours becomes a live cell, as if by reproduction
\end{itemize}
\par
The initial pattern constitutes the seed of the system. The first generation is created by applying the above rules simultaneously to every cell in the seed; births and deaths occur simultaneously, and the discrete moment at which this happens is sometimes called a tick. Each generation is a pure function of the preceding one. The rules continue to be applied repeatedly to create further generations. 

\section{Physarum}

Physarum polycephalum \cite{sun2017physarum}, \cite{mayne2016biology} is a species of order Physarales, subclass Myxogas-tromycetidae, class Myxomecetes, division Myxostelida, commonly known as a true slime mold, that inhabits shady, cool and moist areas. 
\par
Physarum thrives in favorable environmental conditions, particularly when the right combinations of humidity, temperature and nutrient presence are created. If the conditions are not adequate for development, Physarum behaves like a single-celled organism that does not demonstrate organizational skills. In appropriate conditions, it joins together to create particularly efficient filamentary nets in physical distribution.
\subsection{Life cycle}
It is common to refer to the Physarum by the name of his vegetative (resting) life cycle phase, the plasmodium. The Physarum plasmodium is a single yellow cytoplasmic mass that can range in size from a few mm$^2$ to over half a m$^2$. The organism will typically be composed of a network of protoplasmic veins that can contain more than 100,000 nuclei.
\par
It is during this stage that the organism searches for food. Multiple sources state that the plasmodium is both predatory and saprophytic: its natural foodstuffs include fungal spores, bacteria, smaller amoebae and decaying matter, the latter of which may be digested extracellularly through the secretion of enzymes.
\par
If environmental conditions cause the plasmodium to desiccate during feeding or migration, Physarum will have another life cycle phase called the sclerotium. The sclerotium is basically highly resistant desiccated tissue that serves as a dormant stage, protecting Physarum for long periods of time. The organism will assume it if environmental conditions become too
unfavourable. Once favorable conditions resume, a sclerotium can be revert back to a viable plasmodium that reappears to continue its quest for food.
\par
As the food supply runs out, the plasmodium stops feeding and begins its reproductive phase. Stalks of sporangia form from the plasmodium. It is within these structures that meiosis occurs and spores are formed. Sporangia are usually formed in the open so that the spores they release will be spread by wind currents.
\par
Spores can remain dormant for years if need be. However, when environmental conditions are favorable for growth, the spores germinate and release either flagellated or amoeboid swarm cells (motile stage). The swarm cells then fuse together to form a new plasmodium. 

